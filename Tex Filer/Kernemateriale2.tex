\begin{document}
En vigtig faktor i fremstilling af elektromagneter er materialet. Ud fra materialet bestemmes, elektromagnetens magnetiake evne, hvori materialets permabiliteten bestemmer hvor det kan lede et magnetisk felt. Permeabiliteten kan deles op i tre underemner, ferromagnetisk, paramagnetisk og diamagnetisk. \\
I diamagnetiske materialer er magnetiseringen ekstremt lille og har en lineær BH-kurve. Ved paramagnetiske materialer er magnetiseringen ligeledes meget lille, men større end diamagnetiske materialer og ligeledes har materialet en lineær BH-kurve. Ferromagnetiske materialer har en stor magnetisering, den magnetiske effekt skyldes herved de uparede elektroner der forekommer i nogle metaller. Ferromagnetiske materialer har en logistisk stigende HB-kurve.\\
  \\
Ferromagnetisk materiale kan nu deles op i to undergrupper, blødt materiale og hårdt materiale. Blødt materiale er er karakteret ved høj permeabilitet, lille hysteresekurve med lille koerciv felt og lille kulstofindhold. Bløde materialer er ofte en legering af jern og silicium eller nikkel.\\
Hårdt materiale er karakteriseret ved mindre permeabilitet end blødt materiale, bred hysteresekurve, stort koerciv felt og højt kulstofindhold. Hårde materialer er ofte en legering af jern med kulstof, aluminium eller wolfram. 
\end{document}
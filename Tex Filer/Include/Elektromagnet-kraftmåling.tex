\newpage 
\subsection{Elektromagnet-kraftmåling}\label{bilag_elektromagnet}
\subsubsection*{Formål}
Formålet med denne journal er at undersøge muligheden for en elektromagnet til at holde bilen på banen i svingene. For at opnå bedst mulige omgangs tid, bør elektromagnet undersøges for om man kan lave en effektiv løsning til at fastholde bilen, så den ikke skrider i sving, med max hastighed. Der laves en prøve magnet, der har pasform til at kunne placeres under bilen. Samtidig skal der testes for hvor meget kræften falder med i forhold til afstand mellem magneten og skinnerne. Derudover testes permanent magneten for hvor stærk den er og sammenlignes herefter med elektromagneten.\\
\\
\subsubsection*{Materiale liste}
\begin{itemize}
	\item Materiale af stål
	\item Kobbertråd af 0.25 mm
	\item Strømforsyning
	\item Pasco Force sensor
	\item Bilens permanent magnet
	\item Magnetisk metal plade
	\item Papir af 0.2 mm tykkelse
	\item Program, PASCO capstone
	\item Computer
\end{itemize}
 
\subsubsection*{Opstilling}

\begin{wrapfigure}{r}{0.5\textwidth}
\includegraphics[scale=0.02, angle=270]{./Graphics/Forsogs_opstilling}
\caption{Forsøgsopstilling for elektromagnet på kreftsensor}
\label{Opstilling}
\end{wrapfigure}

\begin{enumerate}
	\item Der skæres og slibes en magnet af ferromagnetisk materiale 
	\item Der vikles vindinger rund om magneten indtil en bredde på 4 mm er ca. opfyldt, da der er 6 mm bilens undervogn til skinnerne, skal der opnås en måling på 2 mm luftgab.
	\item Force sensoren placeres så den kan måle magnetens kraft når den trækkes af, en metal plade 
	\item En computer med programmet PASCO capstone som kan læse kraft målingerne.\\
 \end{enumerate}
  
Forsøget går ud på at lave en elektromagnet der kan sidde under bilen, derfor laves en test magnet af 660 vindinger da der ikke kan være så flere vindinger. En kraftmåler opstilles så den kan måle begge elektromagneters ender, der sættes 14 V over elektromagneten og kraften måles nu ved at trække elektromagneten af kraftmåleren. Forsøget gentages nu med et stykke papir på 0.2 mm, dernæst 0,4 mm, indtil vi når op på 2mm. Permanentmagneten testes, den sættes på kraftmåleren og der trækkes ned af, hvor målingerne nu er gemt på computeren.



\subsubsection*{Data}
De nedestående data er et førsøg der er gentaget 3 gange og og luftgabet er lavet med flere stykker 0.2 mm papir, resultaterne er i Newton. \\

\begin{tabular}{|c|c|c|c|c|c|c|c|c|c|c|c|}
\hline 
Luftgab & 0 & 0.2 & 0.4 & 0.6 & 0.8 & 1 & 1.2 & 1.4 & 1.6 & 1.8 & 2 \\ 
\hline 
Forsøg 1 & 1.345 & 1.320 & 1.090 & 1.100 & 1.090 & 1.130 & 1.060 & 1.090 & 0.930 & 0.970 & 0.960 \\ 
\hline 
Forsøg 2 & 1.570 & 1.150 & 1.350 & 1.320 & 1.120 & 1.050 & 1.000 & 1.060 & 1.040 & 0.950 & 0.870 \\ 
\hline 
Forsøg 3 & 1.000 & 1.440 & 1.350 & 1.100 & 1.200 & 1.200 & 1.200 & 1.100 & 1.000 & 0.930 & 0.980 \\ 
\hline 
Gennemsnit & 1.305 & 1.303 & 1.263 & 1.173 & 1.137 & 1.127 & 1.087 & 1.083 & 0.990 & 0.950 & 0.937 \\ 
\hline 
\end{tabular}
\\
De nedestående data er fra målinger permanetmagneten, som er lavet så samme vis som elektromagnet\\

\begin{tabular}{|c|c|c|c|c|c|c|c|c|c|c|c|}
\hline
Luftgab & 0 & 0.2 & 0.4 & 0.6 & 0.8 & 1.0 & 1.2 & 1.4 & 1.6 & 1.8 & 2 \\
\hline
Forsøg 1 & 4.20 & 4.82 & 3.34 & 4.17 & 4.12 & 4.14 & 3.25 & 2.97 & 2.18 & 2.52 & 1.65 \\
\hline
Forsøg 2 & 4.38 & 4.65 & 4.74 & 4.90 & 4.40 & 4.23 & 3.41 & 2.38 & 2.30 & 2.18 & 1.31 \\
\hline
Forsøg 3 & 4.67 & 4.95 & 4.37 & 4.79 & 4.12 & 3.44 & 3.30 & 1.96 & 1.85 & 1.65 & 1.48 \\
\hline 
Gennemsnit & 4.42 & 4.81 & 4.15 & 4.62 & 4.21 & 3.94 & 3.32 & 2.44 & 2.11 & 2.12 & 1.48 \\
\hline
\end{tabular} 

\subsubsection*{Resultater}
\begin{figure}[h!]
\center
\includegraphics[scale=0.3]{./Graphics/Graf_Elektromagnet_resultater}
\caption{Data sættet for elektromagnet behandlet i graf}
\label{Elektromagnet}
\end{figure}

Her ligger punkterne over data sættet, hvor de røde punkter er data for alle forsøgene og de blå er gennemsnittet af disse forsøg. Vi får nu at der er en kraft på 0.9367 N, ved luftgabet på 2 mm.

\subsubsection*{Udregninger}
Vi udregner hvordan elektromagneten ville virke under ideele forhold, det vil sige hvis, der ikke var nogen flux spredning i elektromagneten. For udregningerne begynder antages det at materialet er en negering af jern, som ligger mellem 300-500 relativ permeabilitet.\\

$H_{j}l_{j}+2H_{g}x=IN$ \\

$H_{j}=\frac{B_{j}}{\mu_{0}\mu_{r}},\,H_{g}=\frac{B_{g}
}{\mu_{0}}$\\

Ved at antage at der ikke er nogen spredning i luftgabet er $B_{g}=B_{j}$ som nu giver:\\

$\frac{B_{g}}{\mu_{0}\mu_{r}}*l_{j}+2*\frac{B_{g}}{\mu_{0}}=IN$\\

$B_{g}$ isoleres og nu har vi ligningen:\\

$B_{g}(x)=\frac{\mu_{0}IN}{({\frac{l_{j}}{\mu_{r}}}+2x)}$\\

Da elektromagneten er en stål blanding vil permeabiliteten ligge mellem 300-500, derfor regnes $B_{g}$ nedre og øvre grænse.\\

$B_{g,Min}(x)=\frac{1.256*10^{-6}*1.4*660}{\frac{0.07}{300}+2*0.002}=0.27$\\

$B_{g,Max}(x)=\frac{1.256*10^{-6}*1.4*660}{\frac{0.07}{500}+2*0.002}=0.28$\\

Her kan vi se at forskellen i permeabiliteten ikke har den store betydning når det kommer til udregning af magnetisk induktion (B). Magnetisk feltstyrke (H). 
De næste beregninger gælder for luftgab:\\

$U_{B}=\frac{1}{2\mu_{0}}*\frac{(\mu_{0}IN)^{2}}{({\frac{l_{j}}{\mu_{r}}}+2x)^{2}}*A_{j}*x$ \\

Den magnetiske kraft kan findes ud fra $F_{magn,x}=+\frac{dU_{B}}{dx}$ \\
Hvilket giver:\\
$F_{magn,x}=A_{j}*\frac{(\mu_{0}IN)^{2}}{2\mu_{0}}*\frac{1}{(\frac{l_{j}}{\mu_{r}}+2x)^{2}}+x*(-2)*\frac{2}{(\frac{l_{j}}{\mu_{r}}+2x)^{3}}$\\

Eller ved at reducere udtrykket:\\
$F_{magn}(x)=-{\frac{B^{2}_{g}}{2\mu_{0}}}* {A_{j}}* (\frac{4x}{{\frac{l_{j}}{\mu_{r}}}+2x}-1) $\\

Nu kan kraften ideelt set regnes ud:\\
$ F_{magn,Min}=-\frac{0.27^{2}}{2*1.256*10^-6}*4.5*10^-5*(\frac{4*0.002}{\frac{0.07}{300}+2*0.002}-1)=-1.1619N $\\

$ F_{magn,Min}=-\frac{0.28^{2}}{2*1.256*10^-6}*4.5*10^-5*(\frac{4*0.002}{\frac{0.07}{500}+2*0.002}-1)=-1.309N $\\

Minus tegnet indikere at kraften virker nedad i samme retning som tyngdekraften.\\

\subsubsection*{Konklusion}
Ud fra data og beregninger kan man hurtigt konkludere at elektromagneten ikke er ideel, da der er et tab på $0.2249-0.372$N ved en afstand på 0.2 mm fra skinnerne. Tabet er blandt andet fluxspredning og fluxfringing. Derudover kan man se på daterne fra permanentmagneten at de falder og stiger hurtigt og er meget varierende, men som udgangspunkt går vi ud fra at de er rigtige. Permanentmagneten er 0.543N stærkere end elektromagneten ved de 0.2 mm afstand fra skinnerne.  



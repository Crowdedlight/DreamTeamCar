\subsection{Beregning af gearing}
\label{beregn_gear}

For at udregne hvor langt hjulet drejer på en puls og hvor mange pulse der skal til for at hjulet drejer en hel omgang skal gearing udregnes. \\
Dette udregnes ved at kigger på de 2 tandhjul som er henholdsvis gear 1 og 2. Der kigges på begge tandhjul og deres tænder tælles. Ved at dividerer værdierne fås forholdet mellem tandhjulene. \\
Dette har givet følgende forhold:
\begin{align*}
Gear1 = 10 / 14 = 0,7143 \\
Gear2 = 9 / 127 = 0,3333
\end{align*}
Ligges gearene sammen:
\begin{align*}
Gear1*Gear2 = 0,7143 * 0,3333 = 0,2381
\end{align*}

Dette betyder at når motoren har drejet en omgang så har hjulet drejet \(0,2381\) omgange.
Når sensoren giver en puls, har hjulet derfor kørt: \(\frac{0,2381}{4} = 0,0595\) \\
Da hjulet er 8.5 cm i omkreds så svarer 1 puls til: \(0,0595*8,5 = 0,50595\)

\subsubsection{Periode tid ved 4 m/s}
\label{periode_4ms}
Her udregnes periode tiden for signalet ved en hastighed på 4 m/s:
\begin{align*}
\frac{4m/s}{0,085m} = 47,1 RPS
\end{align*}

Når hjulet har kørt en omgang har motoren altså kørt:
\begin{align*}
(\frac{27}{9}) * (\frac{14}{10}) = 4,2
\end{align*}

Så motoren kører: \(47,1 RPS * 4,2 = 197,82 RPS = 11869,2 RPM \) \\
Så der er: \(197,82 RPS *4 = 791,28 PPS \) \\
\(1/791,28 = 0,001265 sekunder \) \\
Dette giver altså en periode tid på: \(1,26 ms\)

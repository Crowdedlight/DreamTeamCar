\subsection{Beregning af gearing}
\label{beregn_gear}

For at udregne hvor langt hjulet drejer på en puls og hvor mange pulser der skal til for at hjulet drejer en hel omgang skal gearing udregnes. \\
Dette udregnes ved at kigge på de fire tandhjul, som er henholdsvis gear 1 og 2. Der kigges på begge gearing og deres tænder tælles. Ved at dividere værdierne fås forholdet mellem tandhjulene. \\
Dette har givet følgende forhold:
\begin{align*}
Gear1 = 10 / 14 = 0,7143 \\
Gear2 = 9 / 127 = 0,3333
\end{align*}
Ligges gearene sammen:
\begin{align*}
Gear1*Gear2 = 0,7143 * 0,3333 = 0,2381
\end{align*}

Dette betyder at når motoren har drejet en omgang så har hjulet drejet \(0,2381\) omgange. \\
Når sensoren giver en puls, har hjulet derfor drejet: \(\frac{0,2381}{4} = 0,0595 omgange\) \\
Da hjulet er 8,5cm i omkreds så svarer 1 puls til: \(0,0595*8,5cm = 0,50595 cm\)

\subsubsection{Periodetid ved 4 m/s}
\label{periode_4ms}
Her udregnes periode tiden for signalet ved en hastighed på $4 m/s$:
$\frac{4m/s}{0,085m} = 47,1 RPS$ \footnote{Se ordliste i afsnit \ref{ordliste}}

Når hjulet har kørt en omgang har motoren altså kørt: $(\frac{27}{9}) * (\frac{14}{10}) = 4,2$ \\
Så motoren kører: \(47,1 RPS * 4,2 = 197,82 RPS = 11869,2 RPM \) \\
Så der er: \(197,82 RPS *4 = 791,28 PPS \) \footnote{Se ordliste i afsnit \ref{ordliste}} \\
\(1/791,28 = 0,001265 s \) \\
Dette giver altså en periodetid på: \(1,26 ms\) ved en hastighed på 4 $m/s$

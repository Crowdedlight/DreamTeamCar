\section{Omdrejningssensor - Måling af afstand}
\label{afstandmaal}
I dette afsnit beskrives, hvordan bilen måler, hvor langt den har kørt. Her benyttes den samme sensor, som bruges til udregning af farten. Læs mere i afsnit \ref{fartmål}. \\
Afstanden bilen har kørt bruges til at mappe banen. Hvilket beskrives i afsnit \ref{mapning}.

\subsection{Afstand Hardware - TCST1230}
\label{afstandmaal_hardware}
Da sensoren er den samme, som benyttes til måling af fart, vil dens hardware ikke blive beskrevet her. Der henvises i stedet til afsnit \ref{fartmål_hardware} \\
Den eneste ændring på hardwaren er at sensorens udgangssignal til microcontrolleren er forbundet med indgang ”PB1”.

\newpage
\subsection{Afstand Software}
\label{afstandmaal_software}
Hver gang sensoren sender en puls til ”PB1” på microcontrolleren vil Timer1 inkrementere med en. Timer1 er valgt da det er en 16bit timer og derfor kan der køres 327,7 meter før timeren giver overflow. Udregningen er som følger: 
\begin{align*}
16bit = 2^{16} * 0,5cm (pr. puls) = 32768 cm = 327,7 m
\end{align*}
Dette er rigeligt til formålet, da tælleren vil blive nulstillet efter hver lige strækning. \\

Antallet af pulser på en hjulomdrejning er \(16,8\). Ved at tælle antallet af pulser kan det udregnes hvor langt bilen har kørt.  For udregning af antal pulser pr. omdrejning: se afsnit \ref{beregn_gear}. \\

Timer1 er opsat så den kører på en ekstern klokfrekvens. Den eksterne klokfrekvens er signalet fra omdrejningstælleren. Timeren er sat til at trigge på rising edge\footnote{Se ordliste i afsnit \ref{ordliste}}. 


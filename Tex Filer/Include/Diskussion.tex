\section{Diskussion}
Ved stregsensoren var det vigtigt, at afstanden til banen var så lille som muligt, for at sikre at den hvide linje bliver detekteret. En forbedring kunne være en komparator, for at trigger punktet kunne indstilles til brugerens ønske.\\
Omdrejningssensoren kunne forbedres ved at montere en skive med flere vifter. Endvidere kunne der bruges et filtersystem så outputtet ville være lettere for microcontrolleren at detektere. Man kunne også have monteret to omdrejningssensorer. På denne måde ville det være muligt at bestemme hvilken vej akslen drejer. Dette kunne være en ekstra sikkerhed i forbindelse med opbremsningen. De to sensorer ville også give mulighed for at måle den kørte længde med øget præcision.\\
Lowpassfilteret som sidder mellem accelerometeret og microcontrolleren kunne højest sandsynlig forbedres. Cut-Off frekvensen som på nuværende tidspunkt er på omkring 1500 Hz skulle have været tilpasset vores behov. Igen kunne man have anvendt komparatorer. Disse kunne have været koblet til interrupt pins på microcontrolleren. På denne måde ville det ikke være nødvendigt at læse accelerometeret hele tiden.\\
Ud fra vores beregninger ville elektromagneten medføre en forringelse af omgangstiden. Ydermere stemmer beregningerne ikke overens med kraftmålingerne af elektromagneten. Det er svært at vide hvorfor at denne forskel er der. Forskellen skyldes bl.a. fluxspredning, fluxfringing, materiale permeabilitet og hvor godt magneten er vundet. F.eks. er det svært at finde permeabiliten for materialet.\\

\section{Konklusion}
Det er lykkedes at konstruere og programmere en Scalextrix bil så den kan køre autonomt på en vilkårlig bane. Der er implimenteret en kommunikationsprotokol. Denne tillader at bilens hastighed kan justeres. Det er også muligt at bede bilen om forskellige værdier som f.eks. dens hastighed eller centripetalkraften rundt i svingene. Et problem ved protokollen består i at den skal modtage tre bytes. Hvis bilen f.eks. kun modtager to bytes vil hele programmet stoppe indtil de har modtaget sidste byte.  
De optiske sensorer vi benytter lever op til de krav vi stillede dem. Den hvide linje detekteres hver omgang og alle pulserne opfanges af omdrejningssensoren. Accelerometeret er meget støjfølsomt og der skal derfor laves meget efterbehandling af outputtet. Man kunne have brugt mere tid på at behandle dataen, såsom filtre. 
Generelt set er projektet opfyldt, da vi har et færdigt produkt, der er i stand til at køre en vilkårlig bane og øge omgangstiderne gennem intelligent valg af bremsepunkter. 

\newpage
\section{Perspektivering}
\label{perspektivering}

Vi har i dette projekt lært at bygge og sammenkoble sensorer og microcontrollere. Ved hjælp af sensorerne har vi givet bilen en form for kunstig intelligens, idet bilen er i stand til at reagere på fysiske ændringer. Dette kan perspektiveres over til mange andre industrier. At kunne sammenkoble hardware med software og give det mulighed for at reagere på omgivelserne. Dette er grundlaget i næsten alle produkter, der indeholder en kontroller. Det er alt lige fra bilen der reagerer på vejens tilstand og farten, der køres med, til køleskabet som benytter en temperatursensor til at kontrollere temperaturen og sørge for at temperaturen er den samme som den viser på displayet. \\

Specifikt i vores projekt er det nærliggende at perspektivere til et køretøj der skal køre en rute selv så effektivt som muligt. Her kunne tænkes på et robotlager, hvor køretøjer skal køre ting fra A til B af samme eller forskellige ruter. Her skal robotten køre sikkert, men samtidig så effektivt som muligt for at tingene kommer hurtigst fra A til B. Dette kræver en styring der på nogen områder minder om vores. Den skal kunne detektere sving og sørge for at bremse eller sænke hastigheden, så den ikke vælter. Det kunne ligeledes forstilles, at den også skulle kommunikere med de andre robotter gennem Bluetooth eller lignende netværk, for ikke at køre ind i andre og samtidig vide hvor den skal hente/aflevere ting. \\

Generelt egner selve styringen, vi har lavet, til situationer hvor du har behov for, at have noget kørende fra A til B. Især hvis du har behov for at give køretøjet en ny rute, engang i mellem, men stadigvæk vil have at køre optimalt. Da vores styring husker banen vil den køre langsomt de første par omgange, men derefter vil den køre optimalt, da den ved hvor den skal bremse og hvor den skal give gas. 

















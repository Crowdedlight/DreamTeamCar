
\section{Kommunikation mellem computer og bil}
\label{kom_bil}

I opgaveformuleringen stilles visse krav til bilens kommunikation. Det skal være muligt at sende beskeder fra f.eks. en computer til bilen. Denne kommunikation skal som minimum kunne sætte bilens hastighed, og stoppe den igen. For at gøre det skal kommunikationen leve op til en protokol, som er dikteret i opgaveoplægget. Protokollen nødvendiggør at der skal sendes tre bytes hver gang. De tre bytes som sendes af sted har hver deres funktion. Den første byte beskriver typen af besked, anden byte er kommandoen, mens tredje byte er data som skal overføres til bilen. Protokollen kræver kun tre forskellige typer af beskeder, disse tre typer kan ses herunder.

\begin{table}[h]
\begin{tabular}{|c|c|l|}
\hline
Hex Værdi & Type  & \multicolumn{1}{c|}{Bemærkning}                                   \\ \hline
0x55      & SET   & Bruges til at sætte/aktivere en værdi i bilen. Kræver intet svar. \\ \hline
0xAA      & GET   & Bruges til at hente en værdi i bilen.                             \\ \hline
0xBB      & REPLY & Er et svar på en GET besked.                                      \\ \hline
\end{tabular}
\caption{Forskellige typer af beskeder}
\label{forskel_besked}
\end{table}

Typen er efterfulgt af en kommando. Det kræves at der minimum er start og stop kommandoer. Protokollen er dog blevet udvidet så den indeholder en del flere kommandoer. Dette er gjort så det i højere grad er lettere at lave test på bilen. F.eks. er det muligt at måle bilens fart eller accelerationen vinkelret på bilen. I det følgende skema kan de forskellige kommandoer ses.

\begin{table}[h]
\center
\begin{tabular}{|c|c|l|}
\hline
Hex Værdi & Type     &  \multicolumn{1}{c|}{Bemærkning}                                     \\ \hline
0x10      & Start    & Dataværdi mellem 0 og 100                      \\ \hline
0x11      & Stop     & Dataværdi er underordnet                       \\ \hline
0x12      & Register & Dataværdi mellem 0 og 25                       \\ \hline
0x13      & MapH     & MSB af banelængden                             \\ \hline
0x14      & MapL     & LSB af banelængden                             \\ \hline
0x15      & Acc      & Returnerer en værdi mellem 0 og 255            \\ \hline
0x16      & Speed    & Periode tiden fra Sensor                       \\ \hline
0xBB      & Error    & Sendes tilbage 3 gange ved fejl                \\ \hline
\end{tabular}
\caption{Forskellige kommandoer}
\label{forskel_kommando}
\end{table}

\textbf{Start} kommandoen bruges til at sætte bilens hastighed. Hastigheden sendes som et procenttal mellem 0-100\%. Dette tal skal konverteres til hex inden afsendelse. Hexværdien skal derfor være mellem 0x00-0x64. Hvis en værdi højere end 0x64 sendes, skal bilen ignorere beskeden, da der er tale om en ugyldig besked. \\

\textbf{Stop} kommandoen bruges til at stoppe bilen. Det er ikke afgørende hvilken data værdi som sendes af sted med stop kommandoen, dog skal der i alt sendes tre bytes af sted, da bilen forventer dette. Man kan således ikke blot sende 0x55 0x11. \\ 

\textbf{Register} kommandoen er indført så der lettere kan udføres debugging. Kommandoen efterfølges med en værdi mellem 20 og 25. Hvis der f.eks. skrives ”Get Register 25” svarer bilen tilbage med Reply Register og værdien i register 25. I dette tilfælde befinder bilens fart sig i register 25. Dette muliggør at man f.eks. kan tjekke om bilen har modtaget den set besked som skulle starte bilen. \\

\textbf{MapH} og \textbf{MapL} er tælleregistrer. Disse to registre udgør tilsammen et 16-bit register som bruges til at tælle antal pulses fra wheelspeedsensoren. På denne måde er det muligt at måle hvor langt bilen har kørt. Når den hvide linje krydses sendes værdierne fra disse registre til f.eks. computeren, herefter nulstilles registrene. Dette blev implementeret så der kunne laves test på hvor konsistent wheelspeedsensoren er. Det 16-bit register tillader en teoretisk max bane længde på……………. Dette kan der læses nærmere om i bilag \todo{Hvilket bilag?} \\

\textbf{Acc} kommandoen bruges til at aflæse ADC’en som er koblet til accelerometeret. På lige strækninger er denne værdi omkring 128…………………. \\

\textbf{Speed} kommandoen anvendes hvis man ønsker at måle bilens fart. Denne kommando er blevet implementeret for at kunne debugge i den del af programmet som sørger for at bilen kører med konstant hastighed. ………………………. \\


\textbf{Error} er svaret på alle ikke kendte beskeder. Dvs. hvis man sender en ikke gyldig kommando til bilen, så svarer den tilbage med 0xBB 0xBB 0xBB. \\

ATMega32 chippen indeholder et USART modul. USART er et initialord for universal synchronous/asynchronous receiver/transmitter. I dette projekt bruges modulet dog bare som UART(universal asynchronous receiver/transmitter). Fordelen ved at benytte asynkron dataoverførsel er at det ikke er nødvendigt med et fælles kloksignal. Da der ikke er et fælles kloksignal skal hver dataoverførsel indledes med et startbit, efterfulgt af 5-9 data bits og beskeden skal slutte med 0-2 stopbits. Det er også muligt at afsende et parity bit. Parity bittet kan bruges til at tjekke om der er fejl i det data som er modtaget. I dette projekt anvendes et start bit, 8 data bits og 1 stop bit. De otte data bits sendes af sted som et hex tal mellem 0x00 og 0xFF. Beskederne til bilen sendes trådløs fra f.eks. en computer via bluetooth. \\
